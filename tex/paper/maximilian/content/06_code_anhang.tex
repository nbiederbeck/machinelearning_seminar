\newpage
\section{Anmerkung zum Code}
\label{sec:06_code_anhang}
\begin{wrapfigure}{l}{0.4\textwidth}
\dirtree{%
.1 weatherpi.
.2 setup.py.
.2 predictions.
.3 preprocessing.
.4 cutter.py.
.4 examine\_data.py.
.4 prepro\_data.py. 
.4 prepro\_rf.py. 
.4 to\_do.pkl.
.3 models.
.4 randomforest.py.
.4 neuralnet.py.
.4 evaluate.py.
.3 pipeline.
.4 link\_data.
.4 pipeline.py.
.2 weatherpi.
.3 Sensor.
.3 TelegramBot.
.4 label.pkl.
.4 label\_handler.py.
.4 \ldots .
.2 documents .
.2 \ldots .
}
\end{wrapfigure}
Das Projekt wird mittels der Versionskontrolle \texttt{GitHub} von den Inhabern
Maximilian Sackel
und Noah Biederbeck gepflegt. 
Der Code wurde überwiegend in den Programmiersprachen Python und die
Dokumentation in \LaTeX verfasst.
Prinzipiell teilt sich das WeatherPi Projekt in zwei Prozesse auf. 
Zu diesen zählt die Datennahme sowie das Labeln der erhobenen Daten und das
erstellen der Vorhersagen.
Die mittels dem \texttt{TelegramBot} zugeteilten Klassenlabel finden sich zu den
dazugehörigen Bildern in der \texttt{label.pkl}.
In dem Ordner \texttt{predictions} finden sich alle Scripte die dem Maschinellen
Lernen dienen.
Die Dateinamen entsprechen dabei ihrer Funktionen.
Zum Ausführen der Dateien müssen alle benötigten Biblotheken mit 
\verb|pip install -r requirements.txt| und das Projekt mit 
\verb|pip install -e .| im weatherpi Projekt installiert werden. 
Zum Ausführen des Codes auf ein Teil des Datensatzes kann der Punkt Evaluieren
ind der \texttt{pipeline.py} ausgeführt werden.

Auf Anfrage kann Einsicht in den Code gewährt werden.
Falls Fehler gefunden werden bitte wir Sie die Email
\href{mailto:maximilian.sackel@gmx.de}{\texttt{maximilian.sackel@gmx.de}} zu 
kontaktieren, damit diese verbesser werden können.
Über Anregungen und Verbesserungsvorschläge freuen wir uns ebenfalls.
