\newpage
\section{Anmerkung zum Code}
\label{sec:06_code_anhang}
\begin{wrapfigure}{l}{0.4\textwidth}
\dirtree{%
.1 weatherpi.
.2 setup.py.
.2 predictions.
.3 preprocessing.
.4 cutter.py.
.4 examine\_data.py.
.4 prepro\_data.py. 
.4 prepro\_rf.py. 
.4 to\_do.pkl.
.3 models.
.4 randomforest.py.
.4 neuralnet.py.
.4 evaluate.py.
.2 weatherpi.
.3 Sensor.
.3 TelegramBot.
.4 label.pkl.
.4 label\_handler.py.
.4 \ldots .
.2 documents .
.2 \ldots .
}
\end{wrapfigure}
Das Projekt wird mittels der Versioncontrolle \texttt{GitHub} von den Inhabern
Maximilian Sackel
und Noah Biederbeck gepflegt. 
Der Code wurde ueberwiegend in den Programmiersprachen Python und die
Documentation in \LaTeX verfasst.
Prinzipell teilt sich das WeatherPi Projekt in zwei Prozesse auf. 
Zu diesen zaehlt die Datennahme sowie das Labeln der erhobenen Daten und das
erstellen der Vorhersagen.
Die mittels dem \texttt{TelegramBot} zugeteilten Klassenlabel finden sich zu den
dazugehoerigen Bildern in der \texttt{label.pkl}.
In dem Ordner \texttt{predictions} finden sich alle Scripte die dem Maschinellen
Lernen dienen.
Die Dateinamen entsprechen dabei ihrer Funktionen.
Zu erwaehnen ist das die Funktion \texttt{examine\_data.py} abhängig von den
Modellen ist um Daten welche nicht den Labeln des Datensatz erneut zu
überprüfen.
Dazu wird ein DataFrame erstellt welche dem TelegramBot mitteilt welche Fotos
eine erneute Überprüfung benötigen.
Zum Ausführen der Files muss das Projekt mit \verb|pip install --user -e setup.py| installiert werden. 
Bedauerlichweise funktioniert die \texttt{Makefile} noch nicht richtig, sodass
gewisse Pakete per Hand nachinstalliert werden müssen.


Auf Anfrage kann Einsicht in den Code gewaehrt werden.
Falls Fehler gefunden werden bitte wir Sie die Email
\href{mailto:maximilian.sackel@gmx.de}{\texttt{maximilian.sackel@gmx.de}} zu 
kontaktieren, damit diese gefixt werden koennen.
Ueber Anregungen und Verbesserungsvorschläge freuen wir uns ebenfalls.
