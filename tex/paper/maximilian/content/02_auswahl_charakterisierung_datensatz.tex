\section{Datensatz}
\label{sec:02_Datensatz}

Der Datensatz wurde mittels einer PiCamera (cite) in einem Zeitraum von 3 Wochen
aufgenommen. 
Die Größe des Datensatzes ist beliebig erweiterbar aber wurde aufgrund der
Zeitlichen einschränkung zur evaluation auf \num{4000} Fotos beschränkt. 

Desweiteren spielt die Zeit in der Die Daten aufgenommen werden eine wesentliche
Rolle in der relativen Häufigkeit in der die verschiedenen Wolkentypen in dem
Datensatz repräsentiert sind. 
So sind zum Beispiel gut-Wetter Wolken viel stärker vertreten als die von
Regenwetter. 
Zu den 11 Wolkenklassen die auch die Klasse keine Wolken einschließt zählt
ebenso eine 'schlechte Fotos' Klasse. 
Aufgrund von zufAlligen als auch zeitlichen Ereignissen werden Fotos produziert
anhand derer keine Wolekklassifikation vorgenommen werden kann.
Ein Beispiel dafür ist wenn eine Person das sichtfeld verdeckt oder es Nacht ist
und die Wolkendecke aufgrund fehlender belichtung nicht erkannt werden kann. 

Die Wolkenfotos wurden an zwei voneinander unabhängigen Orten aufgenommen, wobei
Characteristiken der Aufnahmeorte auf den Bildern zu erkennen sind.
Desweiteren stellte sich bei der Aufnahme des datensatzes heraus das eine
Ausrichtung der Kamera der Wetterstation nach Osten sinnvoll ist.
Dadruch laesst sich ein übermäßiges ausleuchten des Bildes durch die Sonne
verhindern und der Kamerasensor wird geschont.
Darüber hinaus stellte sich heraus das der Aufnahmewinkel $\theta$ entscheidend
fuer die Klassifikation ist. 
Bei grossen Aufnahmewinkel ist nur ein kleinerer Teil der Wolkendecke zu sehen
als bei kleineren.
Bei diesen Ausschnitten stellt es sich als schwierig heraus sie der richtigen
klasse zuzuordnen da der Ausschnitt meist meherern Klassen zuzuordnen ist. 
Mit Grösseren Raumwinkeln lässt sich das Problem reduzieren, da dadurch ein
Größerer Teil der Wolkendecke observiert werden kann. 

Die aufgenommenen Daten besitzen a priori kein Label.
Anhand der im Anhang kurze Erklärung der unterschiedlichen Klassen wurden Sie
mittels eines eigen dafür programmierten \texttt{Telegram-Bot} eingeteilt.
Mit hilfe von Freiwilliigen labelnern wurde jedes Foto des Datensatzes drei mal
klassifiziert und anschliessend per mehrheitsentscheid der Zielklasse zugeteilt.
