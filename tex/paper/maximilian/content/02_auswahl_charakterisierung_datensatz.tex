\section{Datensatz}
\label{sec:02_Datensatz}

Der Datensatz wurde mittels einer PiCamera in einem Zeitraum von 3 Wochen
aufgenommen. 
Die Größe des Datensatzes ist beliebig erweiterbar, aber wurde aufgrund der
zeitlichen Einschränkung zur Evaluation auf \num{4000} Fotos beschränkt. 

\begin{wrapfigure}{l}{0.35\textwidth}
		\centering
		\vspace{-0.5cm}
		\includegraphics[width=0.3\textwidth]{./build/station_winkel.pdf}
		\caption{Wolkenausschnitt in Abhängigkeit des Observationswinkels.}
		\label{fig:theta}
		\vspace{-0.5cm}
\end{wrapfigure}
Desweiteren spielt die Zeit in der die Daten aufgenommen werden eine 
wesentliche Rolle.
In Abhängigkeit mit der Jahreszeiten ändern sich die relativen Häufigkeit, in 
der die verschiedene Wolkentypen aufgenommen werden können und somit in dem 
Datensatz repräsentiert sind. 
So sind zum Beispiel momentan (\today) Schönwetterwolken viel häufiger
vertreten, als die von Regenwetter. 
Zu den elf Wolkenklassen, die auch die Klasse \textit{keine Wolken} einschließt, 
zählt ebenso eine \textit{schlechtes Foto} Klasse. 
Aufgrund von zufälligen als auch zeitlichen Ereignissen werden Fotos produziert
die nicht Klassifiziert werden können.
Dies ist beispielsweise der Fall, wenn eine Person das Sichtfeld verdeckt oder 
aufgrund der fehlenden Belichtung bei Nacht die Wolkendecke nicht erkannt wird.

Die Wolkenfotos wurden an zwei voneinander unabhängigen Orten aufgenommen, wobei
Charakteristiken der Aufnahmeorte auf den Bildern zu erkennen sein können.
Desweiteren stellte sich bei der Aufnahme des Datensatzes heraus, dass eine
Ausrichtung der Kamera der Wetterstation nach Norden sinnvoll ist.
Dadurch lässt sich der Kamerasensor schonen, indem ein übermäßiges Ausleuchten
 des Bildes durch die Sonne verhindert wird.

Durch die Ausrichtung der Kamera zum Horizont kann, wie in Abbildung
 \ref{fig:theta} dargestellt, die Größe der observierten Wolkenfläche 
eingestellt werden.
Dadurch, dass die Wolken bei großen Aufnahmewinkeln $\omega$ viel näher als bei
kleinen $\omega$ sind, kann bei gleichbleibender Raumwinkelauflösung $\Omega$
nur ein viel kleinerer Teil der Wolkendecke observiert werden.
Bei diesen Ausschnitten stellt es sich als schwierig heraus, sie der richtigen
Klasse zuzuordnen, da der Ausschnitt meist mehreren Klassen zuzuordnen ist. 

\begin{wrapfigure}{r}{0.35\textwidth}
		\vspace{-1.4cm}
		\centering
		\includegraphics[width=0.35\textwidth]{pictures/telegram.pdf}
		\caption{\href{https://telegram.me/weatherpi_bot}{\texttt{TelegramBot}} zum Labeln der Wolkenfotos.}
		\label{fig:}
		\vspace{-1.0cm}
\end{wrapfigure}
Die aufgenommenen Daten besitzen a priori kein Label und sind auch nicht immer
eindeutig einer Klasse zuzuordnen.
Die Wolkenfotos wurden mittels eines eigen dafür programmierten
\href{https://telegram.me/weatherpi_bot}{\texttt{TelegramBot}} in
die in Abbildung~\ref{fig:classes} aufgeführten Kategorien eingeteilt.
Mit Hilfe von freiwilligen Labelern wurde jedes Foto des Datensatzes drei mal
gelabelt und anschließend per Mehrheitsentscheidung der Zielklasse 
zugeteilt.
Der Arbeitsaufwand wurde mit ca $\num{4000} \, \text{Bilder} \cdot \SI{1}{\hertz}$
abgeschätzt.
Aufgrund von Aussetzern des Bots und der Ladezeiten der Bilder konnte die Rate
auch nach einer Einarbeitungszeit nicht erreicht werden und die pro Bild benötigte Zeit entsprach schätzungsweise
\SI{15}{\second}.
Bei der Evaluation der Modelle stellte sich heraus, dass bei paralleler Nutzung
eine interne Klassenvariable überschrieben wurde, sodass ein Großteil der
Klassifizierten Daten einer falschen finalen Klasse zugeordnet wurden. 
Final steht ein Datensatz von \num{4000} Bildern welche in 10 Wolkenklassen und
einer Klasse mit schlechten Fotos aufgeteilt sind. Die Bilder besitzen eine Dimension 
von \texttt{(1024x768x3)} und liegen im \texttt{JPG}-Format vor.
Sie stehen unter der MIT Lizens und soll allen interessierten 
Datenanalysten zur Verfügung stehen.

\begin{figure}[h]
		\centering
		\begin{subfigure}[b]{0.31\textwidth}
		\begin{center}
				\includegraphics[width=\textwidth]{./pictures/cloudtypes/altocumulus.pdf}
		\end{center}
		\caption{Altocumulus}
		\label{fig:altostratus}
		\end{subfigure}
		\begin{subfigure}[b]{0.31\textwidth}
		\begin{center}
				\includegraphics[width=\textwidth]{./pictures/cloudtypes/altostratus.pdf}
		\end{center}
		\caption{Altostratus}
		\label{fig:altocumulus}
		\end{subfigure}
		\begin{subfigure}[b]{0.31\textwidth}
		\begin{center}
				\includegraphics[width=\textwidth]{./pictures/cloudtypes/cirrocumulus.pdf}
		\end{center}
		\caption{Cirrocumulus}
		\label{fig:cirrocumulus}
		\end{subfigure}
		\begin{subfigure}[b]{0.31\textwidth}
		\begin{center}
				\includegraphics[width=\textwidth]{./pictures/cloudtypes/cirrostratus.pdf}
		\end{center}
		\caption{Cirrostratus}
		\label{fig:cirrostratus}
		\end{subfigure}
		\begin{subfigure}[b]{0.31\textwidth}
		\begin{center}
				\includegraphics[width=\textwidth]{./pictures/cloudtypes/cirrus.pdf}
		\end{center}
		\caption{Cirrus}
		\label{fig:cirrus}
		\end{subfigure}
		\begin{subfigure}[b]{0.31\textwidth}
		\begin{center}
				\includegraphics[width=\textwidth]{./pictures/cloudtypes/cumulus.pdf}
		\end{center}
		\caption{Cumulus}
		\label{fig:Cumulus}
		\end{subfigure}
		\begin{subfigure}[b]{0.31\textwidth}
		\begin{center}
				\includegraphics[width=\textwidth]{./pictures/cloudtypes/nimbostratus.pdf}
		\end{center}
		\caption{Nimbostratus}
		\label{fig:nimbostratus}
		\end{subfigure}
		\begin{subfigure}[b]{0.31\textwidth}
		\begin{center}
				\includegraphics[width=\textwidth]{./pictures/cloudtypes/stratocumulus.pdf}
		\end{center}
		\caption{Stratocumulus}
		\label{fig:stratocumulus}
		\end{subfigure}
		\begin{subfigure}[b]{0.31\textwidth}
		\begin{center}
				\includegraphics[width=\textwidth]{./pictures/cloudtypes/no_clouds.pdf}
		\end{center}
		\caption{keine Wolken}
		\label{fig:no_clouds}
		\end{subfigure}
		\caption{Repräsentative Fotos für die unterschiedlichen Wolkenklassen.
		Dabei können die Wolkenformen stark variieren. Für umfassendere
		Information können bei dem
		\href{https://telegram.me/weatherpi_bot}{\texttt{TelegramBot}} unter dem
		Punkt \textit{label $\rightarrow$ label $\rightarrow$ info} weitere 
		Informationen angefordert werden.}
		\label{fig:classes}
\end{figure}
