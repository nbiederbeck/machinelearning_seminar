\section{Zusammenfassung}
\label{sec:05_zusammenfassung}
Mit Hilfe eines \texttt{TelegramBot}s konnten die selbst aufgenommen
Daten gelabelt werden.
Dabei unterlief ein Fehler, sodass die Klassenzugehörigkeit der Bilder teilweise
falsch eingetragen wurde.
Das neugesteckte Ziel ist eine starke Generalisierung der maschinellen Lernern 
um das Training auf den Daten die einen Missmatch haben zu ermöglichen.
Desweiteren konnte das Verhalten der Algorithmen bei einem Datensatz
mit einem Missmatch studiert werden.
Die Trainierten Modelle wurden dazu verwendet den Missmatch 
suksessive zu verringern.
Desweiteren wurde ein Farbfilter sowie die Möglichkeit einen 
Datensatz per Chat zu labeln geschaffen.
Es wurde durch Varation der Architectur versucht das Netz der Problemstellung
anzupassen und die Parameter geschickt zu wählen.
Die Vorhersagegenauigkeit der Modelle sind mit dem Missmatch antikorelliert was
den Fokus auf eine weitere Bereinigung des Datensatzes legt.
Der Random Forest erweist sich dabei als das Modell welches mit 
einem kleinen Trainingsdatensatz und Missmatchen besser umgehen 
kann.

In Zukunft soll der Datensatz wesentlich erweitert werden um das 
Training des CNN zu verbessern.
Desweiteren wird geprüft ob, bei einem hinreichend großen Datensatz sowie
Genauigkeit, 
Modelle Wolken mittels dem sliding Window verfahren tracken koennen.
Dabei sind besonders die Größenänderung sowie die Bewegungsrichtung von
besonderem Interesse.
Dies könnte zum Beispiel sowohl die Präzesion der Wettervorhersage, als auch 
die Routenberechnung für Segelflieger verbessern. 
Dazu werden lokale Wolkenvorhersagen benötigt, die durch die Methode
möglicherweise geschaffen werden können.
