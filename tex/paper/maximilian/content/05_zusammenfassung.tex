\section{Zusammenfassung}
\label{sec:05_zusammenfassung}
Mit Hilfe eines \texttt{TelegramBot}s konnten die selbst aufgenommen
Daten gelabelt werden.
Dabei unterlief ein Fehler, sodass die Klassenzugehörigkeit der Bilder teilweise
falsch eingetragen wurde.
Das neugesteckte Ziel ist eine starke Generalisierung um den 
Missmatch zu handeln.
Desweiteren konnte das Verhalten der Algorithmen bei einem Datensatz
mit einem Missmatch studiert werden.
Die Trainierten Modelle wurden dazu verwendet den Missmatch 
suksessive zu verringern.
Desweiteren wurde ein Farbfilter sowie die Möglichkeit einen 
Datensatz per Chat zu labeln geschaffen.
Die Vorhersagegenauigkeit der Modelle sind mit dem Missmatch antikorelliert was
den Fokus auf eine weitere Bereinigung des Datensatzes legt.
Der Random Forest erweist sich dabei als das Modell welches mit 
einem kleinen Trainingsdatensatz und Missmatchen besser umgehen 
kann.

In Zukunft soll der Datensatz wesentlich erweitert werden um das 
Training des CNN zu verbessern.
Desweiteren wird geprüft ob bei einem hinreichend großen Datensatz und Accuracy
der Modelle Wolken mittels dem sliding Window verfahren getrackt werden koennen.
Dabei sind besonders die Größenänderung sowie die Bewegungsrichtung von
besonderem Interesse.
Dies könnte sowhol die Präzesion der Wettervorhersage als auch bei
die Routenberechnung des Segelfliegens verbessern. 
Dazu werden lokale Wolkenvorhersagen benötigt, die durch die Methode
möglicherweise geschaffen werden können.
